% !TEX TS-program = latex
% !TEX encoding = UTF-8

\documentclass[unicode,hyperfootnotes=false,xetex,colorlinks=true,nofonts,nobib]{tufte-book} % use larger type; default would be 10pt

%%
% Book metadata
\title{Finale Settings}
\author{Rachael Carlson}
\publisher{rachaelcarlson.com}

%%
% For nicely typeset tabular material
\usepackage{booktabs}

%%
% For graphics / images
\usepackage{graphicx}
\setkeys{Gin}{width=\linewidth,totalheight=\textheight,keepaspectratio}
\graphicspath{{graphics/}}

%%
% The fancyvrb package lets us customize the formatting of verbatim
% environments. We use a slightly smaller font
\usepackage{fancyvrb}
\fvset{fontsize=\normalsize}

%%
% Prints argument within hanging parentheses (i.e., parentheses that take
% up no horizontal space).  Useful in tabular environments.
\newcommand{\hangp}[1]{\makebox[0pt][r]{(}#1\makebox[0pt][l]{)}}

%%
% Prints an asterisk that takes up no horizontal space.
% Useful in tabular environments.
\newcommand{\hangstar}{\makebox[0pt][l]{*}}

%%
% Prints a trailing space in a smart way.
\usepackage{xspace}

% Prints the month name (e.g., January) and the year (e.g., 2008)
\newcommand{\monthyear}{%
  \ifcase\month\or January\or February\or March\or April\or May\or June\or
  July\or August\or September\or October\or November\or
  December\fi\space\number\year
}


% Prints an epigraph and speaker in sans serif, all-caps type.
\newcommand{\openepigraph}[2]{%
  %\sffamily\fontsize{14}{16}\selectfont
  \begin{fullwidth}
  \sffamily\large
  \begin{doublespace}
  \noindent\allcaps{#1}\\% epigraph
  \noindent\allcaps{#2}% author
  \end{doublespace}
  \end{fullwidth}
}

% Inserts a blank page
\newcommand{\blankpage}{\newpage\hbox{}\thispagestyle{empty}\newpage}

\usepackage{units}

% Typesets the font size, leading, and measure in the form of 10/12x26 pc.
\newcommand{\measure}[3]{#1/#2$\times$\unit[#3]{pc}}

%%
% Font configurations
\usepackage{fontspec} % Font selection for XeLaTeX; see fontspec.pdf for documentation
\defaultfontfeatures{Mapping=tex-text} % to support TeX conventions like ``---''
\usepackage{xunicode} % Unicode support for LaTeX character names (accents, European chars, etc)
\usepackage{xltxtra} % Extra customizations for XeLaTeX
\usepackage{ifxetex}
\ifxetex
  \newcommand{\textls}[2][5]{%
    \begingroup\addfontfeatures{LetterSpace=#1}#2\endgroup
  }
  \renewcommand{\allcapsspacing}[1]{\textls[15]{#1}}
  \renewcommand{\smallcapsspacing}[1]{\textls[10]{#1}}
  \renewcommand{\allcaps}[1]{\textls[15]{\MakeTextUppercase{#1}}}
  \renewcommand{\smallcaps}[1]{\smallcapsspacing{\scshape\MakeTextLowercase{#1}}}
  \renewcommand{\textsc}[1]{\smallcapsspacing{\textsmallcaps{#1}}}
\fi
\setmainfont[Numbers=OldStyle]{Libertinus Serif}
\setsansfont[]{Nimbus Sans L}
% \setmonofont{Deja Vu Mono}
\usepackage{etoolbox}

%%
% Use monospaced numbers within the tabular environment
\AtBeginEnvironment{tabular}{\setmainfont[Numbers={Monospaced}]{Libertinus Serif}}

% other LaTeX packages.....
\usepackage{geometry} % See geometry.pdf to learn the layout options. There are lots.
\geometry{letterpaper}

% \usepackage{hyperref}
% \hypersetup{
%     colorlinks,
%     citecolor=black,
%     filecolor=black,
%     linkcolor=red,
%     urlcolor=black
% }

% Generates the index
\usepackage{makeidx}
\makeindex

\begin{document}
\frontmatter

% epigraphs
% I love these epigraphs. I feel that they are perfect for this project as well.
\newpage\thispagestyle{empty}
\openepigraph{%
The public is more familiar with bad design than good design.
It is, in effect, conditioned to prefer bad design, 
because that is what it lives with. 
The new becomes threatening, the old reassuring.
}{Paul Rand%, {\itshape Design, Form, and Chaos}
}
\vfill
\openepigraph{%
A designer knows that he has achieved perfection 
not when there is nothing left to add, 
but when there is nothing left to take away.
}{Antoine de Saint-Exup\'{e}ry}
\vfill
\openepigraph{%
\ldots the designer of a new system must not only be the implementor and the first 
large-scale user; the designer should also write the first user manual\ldots 
If I had not participated fully in all these activities, 
literally hundreds of improvements would never have been made, 
because I would never have thought of them or perceived 
why they were important.
}{Donald E. Knuth}


% Title Page
\maketitle


% copyright page
\newpage
\begin{fullwidth}
  ~\vfill
  \thispagestyle{empty}
  \setlength{\parindent}{0pt}
  \setlength{\parskip}{\baselineskip}
  Copyright \copyright\ \the\year\ \thanklessauthor

  \par\smallcaps{Published by \thanklessauthor}

  \par\smallcaps{https://www.rachaelcarlson.com}

  \par Licensed under the Apache License, Version 2.0 (the
  ``License''); you may not use this file except in compliance with
  the License. You may obtain a copy of the License at
  \url{http://www.apache.org/licenses/LICENSE-2.0}. Unless required by
  applicable law or agreed to in writing, software distributed under
  the License is distributed on an \smallcaps{``AS IS'' BASIS, WITHOUT
    WARRANTIES OR CONDITIONS OF ANY KIND}, either express or
  implied. See the License for the specific language governing
  permissions and limitations under the License.\index{license}

  \par\textit{Alpha Edition, \monthyear}
\end{fullwidth}

%% Contents
\tableofcontents


%% List of figures (we will eventually have some figures to add
% \listoffigures


\listoftables


%% Dedication
\cleardoublepage
~\vfill
\begin{doublespace}
  \noindent\fontsize{18}{22}\selectfont\itshape
  \nohyphenation
  For John Stropes, Joshua Lane, and \\
  Meghan Carlson, without whom I wouldn't.
\end{doublespace}
\vfill
\vfill


\clearpage
\chapter{Introduction}
\newthought{The purpose} of this document is to generate consistency in the
production of finger-style guitar scores.

\mainmatter

\chapter{Document Settings}
\newthought{These are the settings} which are configurable through the
document settings dialog.
\section{\textsc{tab} Clef}
Edit default TAB clef in Clef Designer (nudge `A' and 	`B' down)\\
Font: TeXGyreSchola, Bold, 10pt
\section{Line Weights}
\begin{table}
  \begin{center}
\begin{tabular}[h!]{l l}
  Line & Weight\\\hline
  Barlines & 5 EVPU\\
  Ledger Lines & 4.6 EVPU\\
  Left Half Ledger Line Length & 7 EVPU\\
  Right Half Ledger Line Length & 7 EVPU\\
  Stems & 2.2 EVPU\\
  Crescendos & 4.2 EVPU\\
\end{tabular}
\end{center}
\caption{Line Weights in EVPU}
\end{table}
\section{Ties}
\paragraph{Placement: Over/Inner}
\begin{tabular}{l l l}
Horizontal: & Start: $-1$pt & End: 1\\
Vertical: & Start: 3 & End: 3\\
\end{tabular}
\paragraph{Placement: Under/Inner}
\begin{tabular}{l l l}
Horizontal: & Start: 9.5 & End: $-9.5$\\
Vertical: & Start $-3$ & End: $-3$
\end{tabular}
\chapter{Tablature Slides}
\section{Smart Shape Tool}
Smart Shape Placement\\
Tab Slide\\
Same V, Lines, Pitch Increasing\\
\begin{tabular}{l l l l}
Start Point: & H: 1.5pt & End Point: & H: $-1.5$\\
& V: 4 & & V: $-5.5$\\
\end{tabular}
\chapter{Page Layout}
\section{Margins}
Left, Right: 54pt\\
Top, Bottom: 36pt
\chapter{Text}
\section{Title}
Font: Avenir Next Heavy, 28pt\sidenote{A bug in Finale 25 on Windows
  10 makes it so that you have to type in the name of the font exactly
  in order for the font to appear on screen. To embed the font for
  print you have to Print to PDF. To do so, in the print dialog,
  choose the Microsoft Print to PDF printer.}

\paragraph{Frame Attributes}
Inserted preset text box (editable through score manager), page 1 only\\
Horizontal: Center, 0pt\\
Vertical: Top (Header), 0\\
Position From: Page Margin\\
Position from Edge of Frame: \textbf{checked}

\section{Subtitle}
\label{sec:subtitle}

Font: Avenir Next, Regular, 8pt

\paragraph{Frame Attributes}
\label{sec:frame-attributes}

Inserted preset text box (editable through Score Manager), page 1 only\\
Centered, Top, Page Margin; H: 0, V: $-32$pt\\
Position from edge of frame: \textbf{checked}

\section{Tuning}
\label{sec:tuning}

Font: Pitches, TeXGyreSchola, Regular, 10pt\\
\indent Octave designations: TeXGyreSchola, Regular, 6pt\\
\indent Baseline shift: -1\\
Accidentals: TeXGyreSchola, Regular, 8pt\\
\indent Superscript: 2

\paragraph{Frame Attributes}
\label{sec:frame-attributes-1}

Text box, page 1 only\\
Horizontal: Left, 1\\
Vertical: Top (Header), -36\\
Position from edge of frame: \textbf{checked}

\section{Composer}
\label{sec:composer}

Font: TeXGyreSchola, Regular, 10pt\\

\paragraph{Frame Attributes}
\label{sec:frame-attributes-2}

Inserted preset text box (editable through Score Manager), page 1 only\\
Horizontal: Right; -1pt\\
Vertical: Top (Header), -36 (align with tuning); Arranger -49\\
Position from: Page Margin\\
Position from edge of frame: \textbf{checked}\\

\section{Copyright}
\label{sec:copyright}

Font: TGS, Regular, 8pt (This is a modified version of TeXGyreSchola with Old Style numerals)

\paragraph{Frame Attributes}
\label{sec:frame-attributes-3}

Inserted preset text box (editable through Score Manager), page 1 only\\
Horizontal: Centered, 0\\
Vertical: Bottom (Footer), -2.25\\
Position from: Page Margin\\
Position from edge of frame: \textbf{checked}\\
Justification: Center

\section{Page Number}
\label{sec:page-number}

Font: TGS, Regular, 8pt

\paragraph{Frame Attributes}
\label{sec:frame-attributes-4}

Inserted preset text boxes: [Title] [File Date] [Page Number]/[Total Pages]\\
Attach to: All Pages\\
Horizontal: Right, 0\\
Vertical: Bottom (Footer), -2.25\\
Position From: Page Margin\\
Position from edge of frame: \textbf{checked}

\section{Timecodes}
\label{sec:timecodes}

Font: Avenir Next, Regular, 8pt

\paragraph{Frame Attributes}
\label{sec:frame-attributes-5}

Text box, Measure attached (standard notation)\\
H: 0\\
V: 48\\
Position from edge of frame: \textbf{checked}

\chapter{Text Expressions}
\label{sec:text-expressions}

\section{Tempo}
\label{sec:tempo}

Justification: Left\\
Horizontal Alignment: Start of Time Signature, 0\\
Vertical Alignment: Staff Reference Line, 36

\section{Time Signatures}
\label{sec:time-signatures}

Font: Maestro, bold, 44pt\\
Justification: Left\\
Horizontal Alignment: Start of Time Signature, 0\\
Vertical Alignment: Staff Reference Line, -22.75

\section{Movements}
\label{sec:movements}

Font: TeXGyreSchola, Italic, 9pt\\
Edit Measure Number Regions\\
One Standard notation staff: Left, Left; H: 1.5, V: -66\\
Grand staff: V: ~-142\\
Show on: Top Staff, \textbf{checked}; Exclude Other Staves, \textbf{checked}; Bottom Staff, \textbf{unchecked}

\chapter{Special Tools}
\label{sec:special-tools}

\section{Beam Angle}
\label{sec:beam-angle}

Eighth note stems: -12\\
Sixteenth note stems: -12\\
Beamed eight notes: -8

\section{Stem Length}
\label{sec:stem-length}

Quarter note stems: -12pt

\chapter{Resize Tool}
\label{sec:resize-tool}

\section{Resize System}
\label{sec:resize-system}

Standard Notation: 85\%\sidenote{Click on staff to ensure that you are adjusting the whole staff and not a note.}\\

\noindent Tablature: 90\%

\chapter{Fingerings}
\label{sec:fingerings}

\section{Left-Hand Fingers (Above Staff)}
\label{sec:left-hand-fingers}

% Updated 04/15/2018
Font: TeXGyreSchola, 8pt, \emph{courtesy: 7pt}\\
Enclosure Shape: Circle\\
Line Thickness: 0.44922\\
Height: 10; \emph{courtesy: 9.75}\\
Width: 10; \emph{courtesy: 9.75}\\
Center H: 0\\
V: -0.25\\
Match Height and Width\\
Fixed enclosure size: \textbf{checked}\\
Justification: Center\\
Horizontal: Stem, 2.75; \emph{courtesy: After Clef/Key/Time/Repeat (2.75)}\\
Vertical: Staff Reference Line;\\
First: 12.75pt; Second: 23.75pt; Third: 34.75; Fourth: 45.75

\section{Left-Hand Duration Lines}
\label{sec:left-hand-duration}

When terminated in the same system as its inception, use the \emph{Bracket Tool}.\\
When terminated in a different system than its inception, use the \emph{Line Tool} and make it horizontal.

\paragraph{Elevated Duration Line}
\label{sec:elev-durat-line}

Line Style: Solid; Horizontal, true\\
Thickness: 0.46094\\
End Point Style for elevating duration line one level:\\
\begin{tabular}{l l}
  Start: & End:\\
  Hook, -6.5pt & Hook, -3pt\\
\end{tabular}

\noindent End Point Style for elevating duration line two levels:\\
\begin{tabular}{l l}
  Start: & End:\\
  Hook, -17pt & Hook, -3pt\\
\end{tabular}

\paragraph{Courtesy Parenthesis}
\label{sec:courtesy-parenthesis}

Font: TeXGyreSchola, Regular, 10pt\\
(   ): Three spaces in between each parenthesis\\
Justification: Center\\
Horizontal Alignment Point: After Clef/Key/Time/Repeat: 2.75pt\\
Vertical Alignment Point: Staff Reference Line:\\
\indent First, 12pt; Second, 23pt; Third, 34pt; Fourth, 45pt

\section{Left-Hand Fingerings (Below Staff)}
\label{sec:left-hand-fingerings}

\paragraph{Fourth String}
\label{sec:fourth-string}

Justification: Center\\
Horizontal Alignment: Stem, 2.75pt\\
Vertical Alignment: Staff Reference Line, -17.5pt (quarter), -16.5pt (eighth)

\paragraph{Fifth String}
\label{sec:fifth-string}

Justification: Center\\
Horizontal Alignment: Stem, 2.75pt\\
Vertical Alignment: Staff Reference Line, -26.5pt (quarter), -25.5pt (eighth)

\paragraph{Sixth String}
\label{sec:sixth-string}

Justification: Center\\
Horizontal Alignment: 2.75pt\\
Vertical Alignment: Below Staff Baseline or Entry, -35 (quarter), -34 (eighth)\\

\paragraph{Additional Offsets}
\label{sec:additional-offsets}

Additional Entry Offset:\\
\indent First, -13.75pt; Second, -24.75pt; Third, -35.75pt; Fourth, -46.75pt

\section{Parentheses}
\label{sec:parentheses}

\emph{Note: this is for surrounding a tablature notehead with a parenthesis. This is used when a finger of the left hand is placed on a fret but the right hand does not play the string.}

Font: Avenir Next, Regular, 10pt\\
(   )-three spaces between for single-digit tablature, four spaces for double-digit tablature\\
Justification: Center\\
Horizontal: Stem, 3pt\\
Vertical: Staff Reference Line\\
\indent First, -2.5pt; Second, -11.5pt; Third, -20.5pt; Fourth, -29.5pt; Fifth, -38.5pt; Sixth, -47.5pt

\section{Right-Hand Fingerings}
\label{sec:right-hand-fing}

Font: TeXGyreSchola, Regular, 8pt\\

\paragraph{Positioning: I, M, A}
\label{sec:positioning:-i-m}

Justification: Center\\
Horizontal Alignment: Stem, -5; two-digit numbers -7\\

\begin{table}[h!]
  \centering
  \begin{tabular}{l l r}
    String & Reference & Alignment\\\hline
    First & Staff Reference Line & 2.25\\
    Second & Staff Reference Line & -7.25\\
    Third & Staff Reference Line & -16.5\\
    Fourth & Staff Reference Line & -25.25\\
\end{tabular}
\caption{Vertical Alignment of i, m, a}
\end{table}

\paragraph{Positioning: P}
\label{sec:positioning:-p}

Justification: Center\\
Horizontal Alignment: Stem, -3.75pt\\

\begin{table}[h!]
  \centering
  \begin{tabular}{l l}
    String & Alignment\\\hline
    First & Staff Reference Line, -6.5\\
    Second & Staff Reference Line, -16.5\\
    Third & Staff Reference Line, -24.25; strum, -33\\
    Fourth & Staff Reference Line, -32.75; strum, -41.75\\
    Fifth & Staff Reference Line, -41.75; strum, -50.75\\
    Sixth & Staff Reference Line, -50.5; strum, -55.75\\
  \end{tabular}
  \caption{Vertical Alignment of p}
  \label{tab:p}
\end{table}

\section{Muted Notes}
\label{sec:muted-notes}

Enclosure Shape: Circle\\
Line Thickness: 0.08984\\
Height: 8.5\\
Width: 8.5\\
Center H: 0\\
V: 0.25\\
Match Height and Width\\
Fixed enclosure size: \textbf{checked}

\chapter{Staff Attributes}
\label{sec:staff-attributes}

Notehead font: Avenir Next Medium, 12pt\\

\section{Stems}
\label{sec:stems}

Always down\\
Horizontal Stem Offsets: 0, 0\\
Use vertical offset for notehead end of stems: \textbf{checked}\\
Offset from noteheads: Up, 6.25pt, Down, -6.25pt\\
Use Vertical offset for beam end of stems (offset from staff) \textbf{unchecked}\\


\chapter{Harmonics}
\label{sec:harmonics}

Enter the number for the harmonic node\\
Special Tools > Note Shape Tool\\

\section{Notehead Settings}
\label{sec:notehead-settings}

\paragraph{Positioning}
\label{sec:positioning}

Horizontal: 0\\
Vertical: 1\\
Allow vertical positioning: \textbf{checked}

\paragraph{Font}
\label{sec:font}

Use default notehead font \textbf{unchecked}\\
Zeal 9 plain

\paragraph{Surrounding}
\label{sec:surrounding}

`<' and `>' are separate expressions\\
Font: Zeal 9 plain\\

<:\\
\indent Justification: Center\\
\indent Horizontal Alignment Point: Stem\\
\indent Additional Horizontal Offset: -2.75\\

>:\\
\indent Justification: Center\\
\indent Horizontal Alignment Point: Stem\\
\indent Additional Horizontal Offset: 9\\

\begin{table}[h!]
  \centering
  \begin{tabular}{l l}
    String & Alignment\\\hline
    First & -3pt\\
    Second & -12\\
    Third & -21\\
    Fourth & -30\\
    Fifth & -39\\
    Sixth & -48\\
  \end{tabular}
  \caption{Vertical Alignment of < and >}
\end{table}
\end{document}
%%% Local Variables:
%%% mode: latex
%%% TeX-master: t
%%% End:
