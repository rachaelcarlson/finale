% !TEX TS-program = latex
% !TEX encoding = UTF-8

\documentclass[unicode,hyperfootnotes=false,xetex,colorlinks=true,nofonts,nobib]{tufte-book} % use larger type; default would be 10pt

%%
% Book metadata
\title{Finale Document \\\noindent Preparation}
\author{Rachael Carlson}
\publisher{rachaelcarlson.com}

%%
% For nicely typeset tabular material
\usepackage{booktabs}

%%
% For graphics / images
\usepackage{graphicx}
\setkeys{Gin}{width=\linewidth,totalheight=\textheight,keepaspectratio}
\graphicspath{{graphics/}}

%%
% The fancyvrb package lets us customize the formatting of verbatim
% environments. We use a slightly smaller font
\usepackage{fancyvrb}
\fvset{fontsize=\normalsize}

%%
% Prints argument within hanging parentheses (i.e., parentheses that take
% up no horizontal space).  Useful in tabular environments.
\newcommand{\hangp}[1]{\makebox[0pt][r]{(}#1\makebox[0pt][l]{)}}

%%
% Prints an asterisk that takes up no horizontal space.
% Useful in tabular environments.
\newcommand{\hangstar}{\makebox[0pt][l]{*}}

%%
% Prints a trailing space in a smart way.
\usepackage{xspace}

% Prints the month name (e.g., January) and the year (e.g., 2008)
\newcommand{\monthyear}{%
  \ifcase\month\or January\or February\or March\or April\or May\or June\or
  July\or August\or September\or October\or November\or
  December\fi\space\number\year
}


% Prints an epigraph and speaker in sans serif, all-caps type.
\newcommand{\openepigraph}[2]{%
  %\sffamily\fontsize{14}{16}\selectfont
  \begin{fullwidth}
  \sffamily\large
  \begin{doublespace}
  \noindent\allcaps{#1}\\% epigraph
  \allcaps{#2}% author
  \end{doublespace}
  \end{fullwidth}
}

% Inserts a blank page
\newcommand{\blankpage}{\newpage\hbox{}\thispagestyle{empty}\newpage}

\usepackage{units}

% Typesets the font size, leading, and measure in the form of 10/12x26 pc.
\newcommand{\measure}[3]{#1/#2$\times$\unit[#3]{pc}}

%%
% Font configurations
\usepackage{fontspec} % Font selection for XeLaTeX; see fontspec.pdf for documentation
\defaultfontfeatures{Mapping=tex-text} % to support TeX conventions like ``---''
\usepackage{xunicode} % Unicode support for LaTeX character names (accents, European chars, etc)
\usepackage{xltxtra} % Extra customizations for XeLaTeX
\usepackage{ifxetex}
\ifxetex
  \newcommand{\textls}[2][5]{%
    \begingroup\addfontfeatures{LetterSpace=#1}#2\endgroup
  }
  \renewcommand{\allcapsspacing}[1]{\textls[15]{#1}}
  \renewcommand{\smallcapsspacing}[1]{\textls[10]{#1}}
  \renewcommand{\allcaps}[1]{\textls[15]{\MakeTextUppercase{#1}}}
  \renewcommand{\smallcaps}[1]{\smallcapsspacing{\scshape\MakeTextLowercase{#1}}}
  \renewcommand{\textsc}[1]{\smallcapsspacing{\textsmallcaps{#1}}}
\fi
\setmainfont[Numbers=OldStyle]{ETBembo}
\setsansfont[]{Gill Sans Nova}
\setmonofont[Scale=MatchLowercase]{Libertinus Mono}
\usepackage{etoolbox}

%%
% Use monospaced numbers within the tabular environment
\AtBeginEnvironment{tabular}{\setmainfont[Numbers={Monospaced}]{Libertinus Serif}}

% other LaTeX packages.....
\usepackage{geometry} % See geometry.pdf to learn the layout options. There are lots.
\geometry{letterpaper}

% \usepackage{hyperref}
% \hypersetup{
%     colorlinks,
%     citecolor=black,
%     filecolor=black,
%     linkcolor=red,
%     urlcolor=black
% }

% Generates the index
\usepackage{makeidx}
\makeindex
\begin{document}

\maketitle
\tableofcontents

\chapter{Steps to Produce a New Document}
\label{sec:steps-produce-new}

Before beginning a new Finale document one must ensure that all
elements of the composition are known. These elements include but are
not limited to its form, key signature, time signatures, desired staff
for standard notation, etc. Knowing these elements will ensure that
the preparation of the new document will keep bugs away later
on. Changing these elements after having started the document can
introduce bugs which will either greatly hinder the development of the
document or make it impossible to continue.

\section{Initial Steps}
\label{sec:initial-steps}


\begin{enumerate}
\item New Document
\item Default Document
\item Change Margins
  \begin{itemize}
  \item Settings for MacOS Finale
    \begin{enumerate}
    \item Page Layout Tool
    \item Page Layout $\rightarrow$ Page Margins $\rightarrow$ Edit
      Page Margins...
    \item Top, Bottom: 36pt
    \item Left, Right: 54pt
    \item Change: All Pages
    \item Apply to Parts/Score
      \begin{itemize}
      \item Select Score and Parts
      \end{itemize}
    \end{enumerate}
  \end{itemize}
  \begin{itemize}
  \item Settings for Windows 10 Finale
    \begin{enumerate}
    \item Document $\rightarrow$ Page Format $\rightarrow$ Score
    \item Page Margins
    \item Top, Bottom: 36pt
    \item Left, Right: 54pt
    \end{enumerate}
  \end{itemize}
\item Delete existing measures
  \begin{itemize}
  \item This is necessary in Windows 10 Finale and not necessary in
    MacOS Finale
  \end{itemize}
\item Edit Score Manager Settings
\end{enumerate}

\section{Score Manager}
\label{sec:score-manager}

\textbf{Window $\rightarrow$ Score Manager}

\begin{itemize}
\item Instrument List
  \begin{enumerate}
  \item Standard Notation
    \begin{enumerate}
    \item Change clef for blank staff to desired clef
    \item Add additional standard notation clefs as needed and change
      clef
    \end{enumerate}
  \item Tablature
    \begin{enumerate}
    \item Add Instrument $\rightarrow$ Tablature $\rightarrow$ Guitar [TAB with Stems]
    \item Change clef to serif `TAB'
    \item Notation Style $\rightarrow$ Tablature $\rightarrow$ Settings
      \begin{enumerate}
      \item Change the tuning
        \begin{enumerate}
        \item Edit Instruments
        \item Change pitches of each string to document tuning
          \begin{itemize}
          \item These will be \textsc{midi} note pitches
          \item C\textsubscript{4} = 60
          \end{itemize}
        \end{enumerate}
      \item Default Lowest Fret = 0
      \item Capo Position = 0
      \item Options
        \begin{enumerate}
        \item Show Tuplets \textbf{checked}
        \item Show Clef Only On First Measure \textbf{unchecked}
        \end{enumerate}
      \item Fret Numbers
        \begin{enumerate}
        \item Vertical Offset = -4
        \item Appearance
          \begin{itemize}
          \item Use Letters \textbf{unchecked}
          \item Break Tablature Lines at Numbers \textbf{checked}
          \end{itemize}
        \end{enumerate}
      \end{enumerate}
    \end{enumerate}
  \end{enumerate}
\item File Info
  \begin{enumerate}
  \item Add all pertinent information in each of the respective sections 
  \end{enumerate}
\end{itemize}

\chapter{Measures}
\label{sec:measures}

\begin{enumerate}
\item Edit $\rightarrow$ Add Measures
\item Add the total number of measures for the document
  \begin{itemize}
  \item If the document has multiple movements add the total number for all movements
  \end{itemize}
\end{enumerate}

\chapter{Time and Key Signatures}
\label{sec:time-key-signatures}

\begin{enumerate}
\item Change the time signature for the document
  \begin{itemize}
  \item If the document has multiple changes in time signature, add all of these changes in the appropriate measures
  \end{itemize}
\end{enumerate}
\end{document}  
%%% Local Variables:
%%% mode: latex
%%% TeX-master: t
%%% End:
